\section{Многозначные логики}
\begin{enumerate}
  \item Переведите формулу $x_1 \lor x_2 \rightarrow x_1 \land x_2$ в заданных системах связок:
  \begin{itemize}
    \item (1 балла) $\uparrow$
    \begin{solution}
      На паре показывали, что
      \begin{equation}
        x_1 \lor x_2 \leftrightarrow (x_1 \uparrow x_1) \uparrow (x_2 \uparrow x_2), \quad x_1 \land x_2 \leftrightarrow (x_1 \uparrow x_2) \uparrow (x_1 \uparrow x_2)
      \end{equation}
      Тогда выражение становится равным
      \begin{equation}
        ((x_1 \uparrow x_1) \uparrow (x_2 \uparrow x_2)) \rightarrow ((x_1 \uparrow x_2) \uparrow (x_1 \uparrow x_2))
      \end{equation}
      Выразим импликацию:
      \begin{equation}
        A \rightarrow B = \overline{A} \lor B
      \end{equation}
      Тогда:
      \begin{eqnarray}
        \overline{((x_1 \uparrow x_1) \uparrow (x_2 \uparrow x_2))} \lor ((x_1 \uparrow x_2) \uparrow (x_1 \uparrow x_2)) = \\
        (\overline{((x_1 \uparrow x_1) \uparrow (x_2 \uparrow x_2))} \uparrow \overline{((x_1 \uparrow x_1) \uparrow (x_2 \uparrow x_2))}) \uparrow \\ (((x_1 \uparrow x_2) \uparrow (x_1 \uparrow x_2)) \uparrow ((x_1 \uparrow x_2) \uparrow (x_1 \uparrow x_2))) = \\
        \{[((x_1 \uparrow x_1) \uparrow (x_2 \uparrow x_2)) \uparrow ((x_1 \uparrow x_1) \uparrow (x_2 \uparrow x_2))] \uparrow \\ \left[((x_1 \uparrow x_1) \uparrow (x_2 \uparrow x_2)) \uparrow ((x_1 \uparrow x_1) \uparrow (x_2 \uparrow x_2))\right]\} \uparrow \\ \left[((x_1 \uparrow x_2) \uparrow (x_1 \uparrow x_2)) \uparrow ((x_1 \uparrow x_2) \uparrow (x_1 \uparrow x_2))\right]
      \end{eqnarray}
    \end{solution}
    \item (2 балла) $\downarrow$
    \begin{solution}
      На паре показывали, что
      \begin{equation}
        x_1 \land x_2 \leftrightarrow (x_1 \downarrow x_1) \downarrow (x_2 \downarrow x_2), \quad x_1 \lor x_2 \leftrightarrow (x_1 \downarrow x_2) \downarrow (x_1 \downarrow x_2)
      \end{equation}
      Тогда выражение становится равным
      \begin{eqnarray}
        ((x_1 \downarrow x_2) \downarrow (x_1 \downarrow x_2)) \rightarrow ((x_1 \downarrow x_1) \downarrow (x_2 \downarrow x_2)) = \\
        \overline{((x_1 \downarrow x_2) \downarrow (x_1 \downarrow x_2))} \lor ((x_1 \downarrow x_1) \downarrow (x_2 \downarrow x_2)) = \\
        \{\overline{((x_1 \downarrow x_2) \downarrow (x_1 \downarrow x_2))} \downarrow ((x_1 \downarrow x_1) \downarrow (x_2 \downarrow x_2))\} \downarrow \\ \{\overline{((x_1 \downarrow x_2) \downarrow (x_1 \downarrow x_2))} \downarrow ((x_1 \downarrow x_1) \downarrow (x_2 \downarrow x_2))\} = \\
        \{[((x_1 \downarrow x_2) \downarrow (x_1 \downarrow x_2)) \downarrow ((x_1 \downarrow x_2) \downarrow (x_1 \downarrow x_2))] \downarrow ((x_1 \downarrow x_1) \downarrow (x_2 \downarrow x_2))\} \downarrow \\
        \{[((x_1 \downarrow x_2) \downarrow (x_1 \downarrow x_2)) \downarrow ((x_1 \downarrow x_2) \downarrow (x_1 \downarrow x_2))] \downarrow ((x_1 \downarrow x_1) \downarrow (x_2 \downarrow x_2))\}
      \end{eqnarray}
    \end{solution}
    \item (1 балла) $0, \rightarrow$
    \begin{solution}
      Выразим отрицание, конъюнкцию и дизъюнкцию
      \begin{eqnarray}
        \overline{x_1} \leftrightarrow x_1 \rightarrow 0, \quad x_1 \lor x_2 \leftrightarrow \overline{x_1} \rightarrow x_2 \leftrightarrow (x_1 \rightarrow 0) \rightarrow x_2, \\ x_1 \land x_2 \leftrightarrow \overline{\overline{x_1} \lor \overline{x_2}} \leftrightarrow (x_1 \rightarrow \overline{x_2}) \rightarrow 0 \leftrightarrow (x_1 \rightarrow x_2 \rightarrow 0) \rightarrow 0
      \end{eqnarray}
      Тогда выражение становится равным
      \begin{eqnarray}
        ((x_1 \rightarrow 0) \rightarrow x_2) \rightarrow (x_1 \rightarrow x_2 \rightarrow 0) \rightarrow 0
      \end{eqnarray}
    \end{solution}
    \item (1 балла) $\oplus, \land, 1$
    \begin{solution}
      Выразим отрицание, дизъюнкцию и импликацию
      \begin{eqnarray}
        \overline{x_1} \leftrightarrow 1 \oplus x_1, \quad x_1 \lor x_2 \leftrightarrow x_1 \oplus x_2 \oplus x_1 \land x_2, \\
        x_1 \rightarrow x_2 \leftrightarrow \overline{x_1} \lor x_2 \leftrightarrow 1 \oplus x_1 \oplus x_2 \oplus (1 \oplus x_1) \land x_2 \leftrightarrow 1 \oplus x_1 \oplus x_2 \oplus x_2 \oplus x_1 \land x_2 \leftrightarrow \\
        1 \oplus x_1 \oplus x_1 \land x_2
      \end{eqnarray}
      Тогда выражение становится равным
      \begin{eqnarray}
        (x_1 \oplus x_2 \oplus x_1 \land x_2) \rightarrow x_1 \land x_2 = 1 \oplus (x_1 \oplus x_2 \oplus x_1 \land x_2) \oplus (x_1 \oplus x_2 \oplus x_1 \land x_2) \land (x_1 \land x_2) = \\
        1 \oplus x_1 \oplus x_2 \oplus x_1 \land x_2 \oplus x_1 \land x_1 \land x_2 \oplus x_2 \land x_1 \land x_2 \oplus x_1 \land x_2 \land x_1 \land x_2 = \\
        1 \oplus x_1 \oplus x_2 \oplus x_1 \land x_2 \oplus x_1 \land x_2 \oplus x_2 \land x_1 = \\
        1 \oplus x_1 \oplus x_2 \oplus x_1 \land x_2
      \end{eqnarray}
    \end{solution}
  \end{itemize}
  \item (1 балла) Воспользуйтесь картой Карно с практики и постройте минимальную КНФ для рассмотренной формулы. Для полного решения необходимо указать, какие прямоугольники были объединены.
  \begin{solution}
    Напишем таблицу истинности
    \begin{displaymath}
      \begin{array}{c c|c c|c}
        x_1 & x_2 & y_1 & y_2 & f(x_1, x_2, y_1, y_2)\\
        \hline
        0 & 0 & 0 & 0 & 1\\
        0 & 0 & 0 & 1 & 0\\
        0 & 0 & 1 & 0 & 0\\
        0 & 0 & 1 & 1 & 0\\
        0 & 1 & 0 & 0 & 1\\
        0 & 1 & 0 & 1 & 1\\
        0 & 1 & 1 & 0 & 0\\
        0 & 1 & 1 & 1 & 0\\
        1 & 0 & 0 & 0 & 1\\
        1 & 0 & 0 & 1 & 0\\
        1 & 0 & 1 & 0 & 1\\
        1 & 0 & 1 & 1 & 0\\
        1 & 1 & 0 & 0 & 1\\
        1 & 1 & 0 & 1 & 1\\
        1 & 1 & 1 & 0 & 1\\
        1 & 1 & 1 & 1 & 1\\
      \end{array}
    \end{displaymath}
    \newpage
    Построим карту Карно
    
    \begin{table}[ht!]
      \begin{tabular}{cc|cccc|}
      \cline{3-6}
                                                  &    & \multicolumn{4}{c|}{$y_1y_2$}                                                        \\ \cline{3-6} 
                                                  &    & \multicolumn{1}{c|}{00} & \multicolumn{1}{c|}{01} & \multicolumn{1}{c|}{11} & 10 \\ \hline
      \multicolumn{1}{|c|}{\multirow{4}{*}{$x_1x_2$}} & 00 & \multicolumn{1}{c|}{1}  & \multicolumn{1}{c|}{0}  & \multicolumn{1}{c|}{0}  & 0  \\ \cline{2-6} 
      \multicolumn{1}{|c|}{}                      & 01 & \multicolumn{1}{c|}{1}  & \multicolumn{1}{c|}{1}  & \multicolumn{1}{c|}{0}  & 0  \\ \cline{2-6} 
      \multicolumn{1}{|c|}{}                      & 11 & \multicolumn{1}{c|}{1}  & \multicolumn{1}{c|}{1}  & \multicolumn{1}{c|}{1}  & 1  \\ \cline{2-6} 
      \multicolumn{1}{|c|}{}                      & 10 & \multicolumn{1}{c|}{1}  & \multicolumn{1}{c|}{0}  & \multicolumn{1}{c|}{0}  & 1  \\ \hline
      \end{tabular}
    \end{table}

    Для КНФ нужно склеивать по нулям, и в данном случае получается взять два прямоугольника, каждый из которых с площадью 4 (первый -- сверху справа, второй -- посередине сверху и захватывает еще два нуля посередине снизу), они соответствуют формуле
    \begin{equation}
      (x_1 \lor \overline{y_1}) \land (x_2 \lor \overline{y_2})
    \end{equation}
  \end{solution}
  
\end{enumerate}

\clearpage
