\section{Практика 7}
\begin{enumerate}
    \item (1 балл) Покажите, что формула из лекции является общезначимой:
    $$(\exists x\varphi(x) \rightarrow \forall x\psi(x)) \rightarrow \forall x. \varphi(x) \rightarrow \psi(x)$$
    \item Покажите, что следующие формулы не являются общезначимыми (приведите контрпример — интерпретацию и значения переменных, как на лекции):
    \begin{itemize}
        \item[(a)] (0,5 балла) $\forall x\varphi \lor \psi \leftrightarrow \forall x. \varphi \lor \psi$
        \item[(b)] (0,5 балла) $\exists x\varphi \land \psi \leftrightarrow \exists x. \varphi \land \psi$
        \item[(c)] (0,5 балла) $\forall x\varphi \rightarrow \psi \leftrightarrow \exists x. \varphi \rightarrow \psi$
        \item[(d)] (0,5 балла) $\exists x\varphi \rightarrow \psi \leftrightarrow \forall x. \varphi \rightarrow \psi$
        \item[(e)] (0,5 балла) $\varphi \rightarrow \forall x\psi \leftrightarrow \forall x. \varphi \rightarrow \psi$
        \item[(f)] (0,5 балла) $\varphi \rightarrow \exists x\psi \leftrightarrow \exists x. \varphi \rightarrow \psi$
    \end{itemize}
    \item Постройте ПНФ формул и сколемизируйте результат до $\Pi_1$:
    \begin{itemize}
        \item[(a)] (1 балл) $\exists x\forall y P(x, y) \lor \exists x\forall y Q(x, y)$
        \item[(b)] (1 балл) $\exists x\forall y P(x, y) \rightarrow \exists x\forall y Q(x, y)$
        \item[(c)] (2 балла) $\exists x. \forall y P(y) \rightarrow \overline{\exists z S(z)} \land \exists y Q(x, y)$
    \end{itemize}
\end{enumerate}
\clearpage
