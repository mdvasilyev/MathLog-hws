\section{Практика 7}
\begin{enumerate}
    \item (1 балл) Покажите, что формула из лекции является общезначимой:
    $$(\exists x\varphi(x) \rightarrow \forall x\psi(x)) \rightarrow \forall x. \varphi(x) \rightarrow \psi(x)$$
    \begin{solution}
        \begin{align*}
            &(\exists x\varphi(x) \rightarrow \forall x\psi(x)) \rightarrow \forall x. \varphi(x) \rightarrow \psi(x) \\
            &(\exists x\varphi(x) \rightarrow \forall x\psi(x)) \rightarrow (\forall x. \varphi(x) \rightarrow \psi(x)) \\
            &\sphericalangle\quad \exists x\varphi(x) \rightarrow \forall x\psi(x) = T \\
            &1.\quad \exists x\varphi(x) = T, \quad \forall x\psi(x) = T \Rightarrow  \forall x. \varphi(x) \rightarrow \psi(x) = T \\
            &\text{потому что } \psi(x) \text{ всегда истинно} \\
            &2.\quad \exists x\varphi(x) = F, \quad \forall x\psi(x) = T \Rightarrow  \forall x. \varphi(x) \rightarrow \psi(x) = T \\
            &\text{потому что } \psi(x) \text{ всегда истинно} \\
            &3.\quad \exists x\varphi(x) = F, \quad \forall x\psi(x) = F \Rightarrow  \forall x. \varphi(x) \rightarrow \psi(x) \leftrightarrow \exists x \varphi(x) \rightarrow \psi = F \rightarrow F = T
        \end{align*}
        Таким образом, мы рассмотрели все возможные случаи, когда посылка истинна и показали, что во всех случаях заключение тоже истинно, поэтому данная формула общезначимая.
    \end{solution}
    \item Покажите, что следующие формулы не являются общезначимыми (приведите контрпример — интерпретацию и значения переменных, как на лекции):
    \begin{itemize}
        \item[(a)] (0,5 балла) $\forall x\varphi \lor \psi \leftrightarrow \forall x. \varphi \lor \psi$
        \item[(b)] (0,5 балла) $\exists x\varphi \land \psi \leftrightarrow \exists x. \varphi \land \psi$
        \item[(c)] (0,5 балла) $\forall x\varphi \rightarrow \psi \leftrightarrow \exists x. \varphi \rightarrow \psi$
        \begin{solution}
            Пусть $\varphi \mapsto (x \text{ четное}), \psi \mapsto (x = 2)$. Из этого не следует ($\leftarrow$), что если есть такой $x$ для которого если он четный, то он равен 2, что для любого четного $x$ следует, что он равен 2. Контрпример: $x = 4$.
        \end{solution}
        \item[(d)] (0,5 балла) $\exists x\varphi \rightarrow \psi \leftrightarrow \forall x. \varphi \rightarrow \psi$
        \begin{solution}
            Пусть $\varphi \mapsto (x \text{ является степенью 2}), \psi \mapsto (x < 5)$. Из этого не следует ($\rightarrow$), что если есть такой $x$, который является степенью 2, то он меньше 5, что для любого $x$, являющегося степенью 2 следует, что он меньше 5. Контрпример: $x = 8$.
        \end{solution}
        \item[(e)] (0,5 балла) $\varphi \rightarrow \forall x\psi \leftrightarrow \forall x. \varphi \rightarrow \psi$
        \item[(f)] (0,5 балла) $\varphi \rightarrow \exists x\psi \leftrightarrow \exists x. \varphi \rightarrow \psi$
    \end{itemize}
    \item Постройте ПНФ формул и сколемизируйте результат до $\Pi_1$:
    \begin{itemize}
        \item[(a)] (1 балл) $\exists x\forall y P(x, y) \lor \exists x\forall y Q(x, y)$
        \begin{solution}
            \begin{align*}
                &\exists x\forall y P(x, y) \lor \exists x\forall y Q(x, y) \\
                &\exists x . \forall y P(x, y) \lor \forall y Q(x, y) \\
                &\exists x . \forall y P(x, y) \lor \forall z Q(x, z) \\
                &\exists x . \forall y. P(x, y) \lor \forall z Q(x, z) \\
                &\exists x . \forall y. \forall z . P(x, y) \lor Q(x, z) \\
                &\forall y. \forall z . P(x, y) \lor Q(x, z)
            \end{align*}
        \end{solution}
        \item[(b)] (1 балл) $\exists x\forall y P(x, y) \rightarrow \exists x\forall y Q(x, y)$
        \begin{solution}
            \begin{align*}
                &\exists x\forall y P(x, y) \rightarrow \exists x\forall y Q(x, y) \\
                &\exists x\forall y P(x, y) \rightarrow \exists z\forall u Q(z, u) \\
                &\forall x . \forall y P(x, y) \rightarrow \exists z\forall u Q(z, u) \\
                &\forall x . \exists y . P(x, y) \rightarrow \exists z\forall u Q(z, u) \\
                &\forall x . \exists y . \exists z . P(x, y) \rightarrow \forall u Q(z, u) \\
                &\forall x . \exists y . \exists z . \forall u . P(x, y) \rightarrow Q(z, u) \\
                &\forall x . \forall u . P(x, y) \rightarrow Q(z, u)
            \end{align*}
        \end{solution}
        \item[(c)] (2 балла) $\exists x. \forall y P(y) \rightarrow \overline{\exists z S(z)} \land \exists y Q(x, y)$
        \begin{solution}
            \begin{align*}
                &\exists x. \forall y P(y) \rightarrow \overline{\exists z S(z)} \land \exists y Q(x, y) \\
                &\exists x. \forall y P(y) \rightarrow \forall z \overline{S(z)} \land \exists y Q(x, y) \\
                &\exists x. \forall y P(y) \rightarrow \forall z \overline{S(z)} \land \exists u Q(x, u) \\
                &\exists x. \exists y . P(y) \rightarrow \forall z \overline{S(z)} \land \exists u Q(x, u) \\
                &\exists x. \exists y . P(y) \rightarrow \exists u . \forall z \overline{S(z)} \land  Q(x, u) \\
                &\exists x. \exists y . \exists u . P(y) \rightarrow \forall z \overline{S(z)} \land  Q(x, u) \\
                &\exists x. \exists y . \exists u . P(y) \rightarrow \forall z . \overline{S(z)} \land  Q(x, u) \\
                &\exists x. \exists y . \exists u . \forall z . P(y) \rightarrow \overline{S(z)} \land  Q(x, u) \\
                &\forall z . P(y) \rightarrow \overline{S(z)} \land  Q(x, u)
            \end{align*}
        \end{solution}
    \end{itemize}
\end{enumerate}
\clearpage
