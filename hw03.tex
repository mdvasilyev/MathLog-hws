\section{Исчисление высказываний}
\begin{enumerate}
  \item Покажите, что следующие формулы являются теоремами исчисления высказываний. Приведите дерево вывода (не обязательно вывода типа) и лямбда-терм, имеющий соответствующий тип.
  \begin{itemize}
    \item[(a)] (1 балл) $A \land B \lor C \rightarrow (A \lor C) \land (B \lor C)$
    \begin{solution}
      \hspace{0.01cm}
      \begin{prooftree}
        \AxiomC{$A \land B \vdash A \land B$}
        \UnaryInfC{$A \land B \vdash A$}
        \UnaryInfC{$A \land B \vdash A \lor C$}
        \AxiomC{$C \vdash C$}
        \UnaryInfC{$C \vdash A \lor C$}
        \BinaryInfC{$A \land B \lor C \vdash A \lor C$}
        \AxiomC{$A \land B \vdash A \land B$}
        \UnaryInfC{$A \land B \vdash B$}
        \UnaryInfC{$A \land B \vdash B \lor C$}
        \AxiomC{$C \vdash C$}
        \UnaryInfC{$C \vdash B \lor C$}
        \BinaryInfC{$A \land B \lor C \vdash B \lor C$}
        \BinaryInfC{$A \land B \lor C \vdash (A \lor C) \land (B \lor C)$}
        \UnaryInfC{$\vdash A \land B \lor C \rightarrow (A \lor C) \land (B \lor C)$}
      \end{prooftree}
      Терм, соответствующий данному выводу:
      \begin{equation}
        \lambda x. \text{case } x \text{ of } \{\text{inl } x \rightarrow \text{pair (fst x) (snd x)}; \text{inr } x \rightarrow \text{pair x x}\}
      \end{equation}
    \end{solution}
    \item[(b)] (1 балл) $A \land C \lor B \land C \rightarrow (A \lor B) \land C$
    \begin{solution}
      \hspace{0.01cm}
      \begin{prooftree}
        \AxiomC{$A \land C \vdash A \land C$}
        \UnaryInfC{$A \land C \vdash A \land C \lor B \land C$}
        \UnaryInfC{$A \land C \vdash (A \lor B) \land C$}
        \AxiomC{$B \land C \vdash B \land C$}
        \UnaryInfC{$B \land C \vdash A \land C \lor B \land C$}
        \UnaryInfC{$B \land C \vdash (A \lor B) \land C$}
        \BinaryInfC{$A \land C \lor B \land C \vdash (A \lor B) \land C$}
        \UnaryInfC{$\vdash A \land C \lor B \land C \rightarrow (A \lor B) \land C$}
      \end{prooftree}
      Терм, соответствующий данному выводу:
      \begin{equation}
        \lambda x. \text{case } x \text{ of } \{\text{inl } x \rightarrow \text{pair (fst x) (snd x)}; \text{inr } x \rightarrow \text{pair (fst x) (snd x)}\}
      \end{equation}
    \end{solution}
    \item[(c)] (1 балл) $(A \lor C) \land (B \lor C) \rightarrow A \land B \lor C$
    \begin{solution}
      \hspace{0.01cm}
      \begin{prooftree}
        \AxiomC{$(A \lor B \lor C), C \vdash C$}
        \UnaryInfC{$(A \lor B \lor C) \land C \vdash C$}
        \UnaryInfC{$A \land C \lor C \land B \lor C \land C \vdash C$}
        \UnaryInfC{$A \land B \lor A \land C \lor C \land B \lor C \land C \vdash A \land B \lor C$}
        \UnaryInfC{$(A \lor C) \land (B \lor C) \vdash A \land B \lor C$}
        \UnaryInfC{$\vdash (A \lor C) \land (B \lor C) \rightarrow A \land B \lor C$}
      \end{prooftree}
      Терм, соответствующий данному выводу:
      \begin{eqnarray}
        \lambda x. \text{case fst } x \text{ of } \{\text{inr (fst } x) \rightarrow \text{inr (fst x)}; \text{inl (fst } x) \rightarrow \text{case snd } x \text{ of } \\\{\text{inr (snd } x) \rightarrow \text{inr (snd } x);\text{inl (snd } x) \rightarrow \text{pair (fst } x) (\text{snd }x)\}\}
      \end{eqnarray}
    \end{solution}
    \item[(d)] (2 балла) $((((A \rightarrow B) \rightarrow A) \rightarrow A) \rightarrow B) \rightarrow B$
    \begin{solution}
      \hspace{0.01cm}
      \begin{prooftree}
        \AxiomC{$A, \overline{B} \vdash A$}
        \UnaryInfC{$A \land \overline{B}, \overline{B} \vdash A$}
        \UnaryInfC{$\overline{\overline{A} \lor B}, \overline{B} \vdash A$}
        \UnaryInfC{$\overline{A \rightarrow B}, \overline{B} \vdash A$}
        \AxiomC{$A, \overline{B} \vdash A$}
        \BinaryInfC{$\overline{(A \rightarrow B)} \lor A, \overline{B} \vdash A$}
        \UnaryInfC{$(A \rightarrow B) \rightarrow A, \overline{B} \vdash A$}
        \UnaryInfC{$(A \rightarrow B) \rightarrow A, \overline{B} \vdash \overline{\overline{A}}$}
        \UnaryInfC{$(A \rightarrow B) \rightarrow A, \overline{A} \vdash B$}
        \UnaryInfC{$((A \rightarrow B) \rightarrow A) \land \overline{A} \vdash B$}
        \UnaryInfC{$\overline{\overline{((A \rightarrow B) \rightarrow A)} \lor A} \vdash B$}
        \UnaryInfC{$\overline{((A \rightarrow B) \rightarrow A) \rightarrow A} \vdash B$}
        \AxiomC{$B \vdash B$}
        \BinaryInfC{$\overline{(((A \rightarrow B) \rightarrow A) \rightarrow A)} \lor B \vdash B$}
        \UnaryInfC{$(((A \rightarrow B) \rightarrow A) \rightarrow A) \rightarrow B \vdash B$}
        \UnaryInfC{$\vdash ((((A \rightarrow B) \rightarrow A) \rightarrow A) \rightarrow B) \rightarrow B$}
      \end{prooftree}
      Терм, соответствующий данному выводу:
      \begin{equation}
        \lambda f. f (\lambda g. g (\lambda x . f (\lambda y . x)))
      \end{equation}
    \end{solution}
  \end{itemize}
  \item (1 балл) Покажите, что закон де Моргана является теоремой исчисления высказываний:
  $$\overline{A \lor B} \rightarrow \overline{A} \land \overline{B}$$
\end{enumerate}
\clearpage
