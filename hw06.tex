\section{Алгебраические теории и логика первого порядка}
\begin{enumerate}
    \item (1б.) Опишите двухсортную сигнатуру и теорию коммутативных колец с единицей и модулей над ними.
    \begin{solution}
        $M$ -- модуль, $A$ -- коммутативное кольцо.
        \begin{eqnarray}
            &\Sigma = \langle\{A, M\}, \{*:A \times A \rightarrow A, +:A \times A \rightarrow A, e: A\}\rangle \\
            &(a*b)*m = a*(b*m), \quad \forall a, b \in A, m \in M \\
            &e*m = m, \quad m \in M \\
            &(a+b)*m = a*m+b*m, \quad \forall a, b \in A, m \in M \\
            &a*(m+n) = a*m+a*n, \quad \forall a \in A, m, n \in M
        \end{eqnarray}
    \end{solution}
    \item Выразите следующие свойства в логике первого порядка (запишите их в виде формул):
    \begin{itemize}
        \item[(a)] (1б.) ”Существует ровно два элемента, удовлетворяющих $P(x)$”
        \begin{solution}
            Придумал два решения:
            \begin{eqnarray}
                \exists x. \exists y. \forall z. \overline{(x=y)} \land \overline{(x=z)} \land \overline{(y=z)} \land P(x) \land P(y) \land \overline{P(z)} \\
                \exists x. \exists y. P(x) \land P(y) \land \overline{(x=y)} \land \forall z. P(z) \rightarrow (z = x \lor z = y)
            \end{eqnarray}
        \end{solution}
        \item[(b)] (1б.) ”Существует не менее двух элементов, удовлетворяющих $P(x)$”.
        \begin{solution}
            \begin{equation}
                \exists x. \exists y. P(x) \land P(y) \land \overline{(x=y)} \land \exists z. P(z) \land \overline{(z=x)} \land \overline{(z=y)}
            \end{equation}
        \end{solution}
        \item[(c)] (1б.) ”Существует по крайней мере один, но не более двух элементов, удовлетворяющих $P(x)$”
        \begin{solution}
            По сути это означает, что существует либо ровно 1, либо ровно 2 (пункт а), поэтому
            \begin{equation}
                \exists a(P(a) \land \forall b. P(b) \rightarrow (b = a)) \lor \exists x. \exists y. P(x) \land P(y) \land \overline{(x=y)} \land \forall z. P(z) \rightarrow (z = x \lor z = y)
            \end{equation}
        \end{solution}
        \item[(d)] (1б.) ”Существует не более одного элемента, удовлетворяющего $P(x)$”
        \begin{solution}
            По сути это означает, что существует либо ровно 0, либо ровно 1, поэтому
            \begin{equation}
                \forall c(\overline{P(c)}) \lor \exists a(P(a) \land \forall b. P(b) \rightarrow (b = a))
            \end{equation}
        \end{solution}
    \end{itemize}
    \item В стандартной интерпретации языка элементарной арифметики выразите следующие свойства:
    \begin{itemize}
        \item[(a)] (1б.) $q$ есть частное от деления $a$ на $b$;
        \begin{solution}
            Здесь и далее пользуюсь тем, что наша модель это $(\mathbb{R}, =, +, \cdot, 0, 1)$. Также удобно ввести порядок:
            \begin{equation}
                a \leq b \leftrightarrow \exists x. b = a + x \cdot x
            \end{equation}
            Дальше не очень понятно, что именно имеется в виду. Если имеется в виду $a/b = q$, то подойдет предикат $Q(a,b,q)$
            \begin{equation}
                Q(a,b,q) \rightleftharpoons \overline{(b=0)} \land q \cdot b = a
            \end{equation}
        \end{solution}
        \item[(b)] (1б.) $r$ есть остаток от деления $a$ на $b$;
        \begin{solution}
            Воспользуемся предикатом из предыдущего пункта $Q(a,b,q)$, тогда можно ввести такой предикат $R(a,b,q,r)$
            \begin{equation}
                R(a,b,q,r) \rightleftharpoons Q(a - r,b,q)
            \end{equation}
            Понятно, что разность в нашей системе несложно выражается, поэтому можно вычесть остаток и проверить предикат из предыдущего пункта на результате разности.
        \end{solution}
        \item[(c)] (1б.) $s$ есть НОД $a$ и $b$;
        \begin{solution}
            Воспользуемся предикатом из предыдущего пункта и заведем новый предикат $D(a,b,s)$
            \begin{equation}
                D(a,b,s) \rightleftharpoons R(a,s,p,0) \land R(b,s,q,0) \land \overline{\exists} s'. s \leq s' \land R(a,s',p',0) \land R(b,s',q',0)
            \end{equation}
        \end{solution}
        \item[(d)] (1б.) $t$ есть НОК $a$ и $b$;
        \item[(e)] (1б.) $a$ и $b$ взаимно просты;
        \item[(f)] (1б.) $u$ является степенью тройки.
    \end{itemize}
    Для сокращения записи пользуйтесь полученными ранее предикатами, введя для них вспомогательные символы.
    
\end{enumerate}
\clearpage
