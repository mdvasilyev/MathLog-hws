\section{Исчисление секвенций}
\begin{enumerate}
  \item Постройте вывод секвенций или приведите контрпример. В этом задании необходимо в
  любом случае приложить дерево (для контрпримера достаточно одной ветки, идущей к
  контрпримеру, но не забывайте указывать названия правил, чтобы было понятно, что
  вы имеете ввиду):
  \begin{itemize}
    \item[(a)] (1 балл) $\vdash (((p \rightarrow q) \rightarrow q) \rightarrow q) \rightarrow p$
    \begin{solution}
      \hspace{0.01cm}
      \begin{prooftree}
        \AxiomC{$\vdash (p \rightarrow q) \rightarrow q, p$}
        \AxiomC{$q \vdash p$}
        \RightLabel{$(\rightarrow\vdash)$}
        \BinaryInfC{$((p \rightarrow q) \rightarrow q) \rightarrow q \vdash p$}
        \RightLabel{$(\vdash\rightarrow)$}
        \UnaryInfC{$\vdash (((p \rightarrow q) \rightarrow q) \rightarrow q) \rightarrow p$}
      \end{prooftree}
      Контрпример: $q$ истинно, $p$ ложно.
    \end{solution}
    \item[(b)] (1 балл) $p \rightarrow q \vdash (p \rightarrow r) \rightarrow q \rightarrow r$
    \item[(c)] (1 балл) $p \leftrightarrow q \vdash (p \lor r) \leftrightarrow q \lor r$
  \end{itemize}
  \item Определите правила введения в антецедент и сукцедент следующих связок и покажите,
  что они действительно выражаются через существующие правила:
  \begin{itemize}
    \item[(a)] (1 балл) $A \uparrow B$ (Штрих Шеффера)
    \item[(b)] (1 балл) $A \downarrow B$ (стрелка Пирса) 
    \item[(c)] (1 балл) $A \oplus B$ (исключающее ИЛИ) 
  \end{itemize}
\end{enumerate}
\clearpage
