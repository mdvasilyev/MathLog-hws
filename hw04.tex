\section{Исчисление секвенций}
\begin{enumerate}
  \item Постройте вывод секвенций или приведите контрпример. В этом задании необходимо в
  любом случае приложить дерево (для контрпримера достаточно одной ветки, идущей к
  контрпримеру, но не забывайте указывать названия правил, чтобы было понятно, что
  вы имеете ввиду):
  \begin{itemize}
    \item[(a)] (1 балл) $\vdash (((p \rightarrow q) \rightarrow q) \rightarrow q) \rightarrow p$
    \begin{solution}
      \hspace{0.01cm}
      \begin{prooftree}
        \AxiomC{$\vdash (p \rightarrow q) \rightarrow q, p$}
        \AxiomC{$q \vdash p$}
        \RightLabel{$(\rightarrow\vdash)$}
        \BinaryInfC{$((p \rightarrow q) \rightarrow q) \rightarrow q \vdash p$}
        \RightLabel{$(\vdash\rightarrow)$}
        \UnaryInfC{$\vdash (((p \rightarrow q) \rightarrow q) \rightarrow q) \rightarrow p$}
      \end{prooftree}
      Контрпример: $q$ истинно, $p$ ложно.
    \end{solution}
    \item[(b)] (1 балл) $p \rightarrow q \vdash (p \rightarrow r) \rightarrow q \rightarrow r$
    \begin{solution}
      \hspace{0.01cm}
      \begin{prooftree}
        \AxiomC{$q \vdash p, p, r$}
        \AxiomC{$r, q \vdash p, r$}
        \RightLabel{$(\rightarrow\vdash)$}
        \BinaryInfC{$p \rightarrow r, q \vdash p, r$}
        \AxiomC{$q, p \rightarrow r, q \vdash r$}
        \RightLabel{$(\rightarrow\vdash)$}
        \BinaryInfC{$p \rightarrow q, p \rightarrow r, q \vdash r$}
        \RightLabel{$(\vdash\rightarrow)$}
        \UnaryInfC{$p \rightarrow q, p \rightarrow r \vdash q \rightarrow r$}
        \RightLabel{$(\vdash\rightarrow)$}
        \UnaryInfC{$p \rightarrow q \vdash (p \rightarrow r) \rightarrow q \rightarrow r$}
      \end{prooftree}
      Контрпример: $q$ истинно, $p$ ложно, $r$ ложно.
    \end{solution}
    \item[(c)] (1 балл) $p \leftrightarrow q \vdash (p \lor r) \leftrightarrow q \lor r$
    \begin{solution}
      \hspace{0.01cm}
      \begin{prooftree}
        \AxiomC{Ниже}
        \UnaryInfC{$p, q \vdash (p \lor r) \leftrightarrow q \lor r$}
        \AxiomC{$p \vdash p, q, q, r$}
        \AxiomC{$r \vdash p, q, q, r$}
        \RightLabel{$(\lor\vdash)$}
        \BinaryInfC{$p \lor r \vdash p, q, q, r$}
        \RightLabel{$(\vdash\lor)$}
        \UnaryInfC{$p \lor r \vdash p, q, q \lor r$}
        \AxiomC{$q \vdash p, q, p, r$}
        \AxiomC{$r \vdash p, q, p, r$}
        \RightLabel{$(\lor\vdash)$}
        \BinaryInfC{$q \lor r \vdash p, q, p, r$}
        \RightLabel{$(\vdash\lor)$}
        \UnaryInfC{$q \lor r \vdash p, q, p \lor r$}
        \RightLabel{$(\vdash\leftrightarrow)$}
        \BinaryInfC{$\vdash p, q, (p \lor r) \leftrightarrow q \lor r$}
        \RightLabel{$(\leftrightarrow\vdash)$}
        \BinaryInfC{$p \leftrightarrow q \vdash (p \lor r) \leftrightarrow q \lor r$}
      \end{prooftree}
      Продолжение
      \begin{prooftree}
        \AxiomC{$p, p, q \vdash q, r$}
        \AxiomC{$r, p, q \vdash q, r$}
        \RightLabel{$(\lor\vdash)$}
        \BinaryInfC{$p \lor r, p, q \vdash q, r$}
        \RightLabel{$(\vdash\lor)$}
        \UnaryInfC{$p \lor r, p, q \vdash q \lor r$}
        \AxiomC{$q, p, q \vdash p, r$}
        \AxiomC{$r, p, q \vdash p, r$}
        \RightLabel{$(\lor\vdash)$}
        \BinaryInfC{$q \lor r, p, q \vdash p, r$}
        \RightLabel{$(\vdash\lor)$}
        \UnaryInfC{$q \lor r, p, q \vdash p \lor r$}
        \RightLabel{$(\vdash\leftrightarrow)$}
        \BinaryInfC{$p, q \vdash (p \lor r) \leftrightarrow q \lor r$}
      \end{prooftree}
      Во всех ветках уперлись в противоречия.
    \end{solution}
  \end{itemize}
  \item Определите правила введения в антецедент и сукцедент следующих связок и покажите,
  что они действительно выражаются через существующие правила:
  \begin{itemize}
    \item[(a)] (1 балл) $A \uparrow B$ (Штрих Шеффера)
    \begin{solution}
      \hspace{0.01cm}
      Сукцедент:
      \begin{prooftree}
        \AxiomC{$A, B \vdash$}
        \RightLabel{$(\land\vdash)$}
        \UnaryInfC{$A \land B \vdash$}
        \RightLabel{$(\vdash\neg)$}
        \UnaryInfC{$\vdash \overline{A \land B}$}
        \UnaryInfC{$\vdash A \uparrow B$}
      \end{prooftree}
      Антецедент:
      \begin{prooftree}
        \AxiomC{$\vdash A$}
        \AxiomC{$\vdash B$}
        \RightLabel{$(\vdash\land)$}
        \BinaryInfC{$\vdash A \land B$}
        \RightLabel{$(\neg\vdash)$}
        \UnaryInfC{$\overline{A \land B} \vdash$}
        \UnaryInfC{$A \uparrow B \vdash$}
      \end{prooftree}
      Итоговые правила формируются как обычно: хотим все то, что сверху и получаем все, что снизу.
    \end{solution}
    \item[(b)] (1 балл) $A \downarrow B$ (стрелка Пирса) 
    \begin{solution}
      \hspace{0.01cm}
      Сукцедент:
      \begin{prooftree}
        \AxiomC{$B \vdash$}
        \AxiomC{$A \vdash$}
        \RightLabel{$(\lor\vdash)$}
        \BinaryInfC{$A \lor B \vdash$}
        \RightLabel{$(\vdash\neg)$}
        \UnaryInfC{$\vdash \overline{A \lor B}$}
        \UnaryInfC{$\vdash A \downarrow B$}
      \end{prooftree}
      Антецедент:
      \begin{prooftree}
        \AxiomC{$\vdash A, B$}
        \RightLabel{$(\vdash\lor)$}
        \UnaryInfC{$\vdash A \lor B$}
        \RightLabel{$(\neg\vdash)$}
        \UnaryInfC{$\overline{A \lor B} \vdash$}
        \UnaryInfC{$A \downarrow B \vdash$}
      \end{prooftree}
      Итоговые правила формируются как обычно: хотим все то, что сверху и получаем все, что снизу.
    \end{solution}
    \item[(c)] (1 балл) $A \oplus B$ (исключающее ИЛИ)
    \begin{solution}
      \hspace{0.01cm}
      Сукцедент:
      \begin{prooftree}
        \AxiomC{$\vdash A, B$}
        \RightLabel{$(\vdash\lor)$}
        \UnaryInfC{$\vdash A \lor B$}
        \AxiomC{$A, B \vdash $}
        \RightLabel{$(\land\vdash)$}
        \UnaryInfC{$A \land B \vdash $}
        \RightLabel{$(\vdash\neg)$}
        \UnaryInfC{$\vdash \overline{A \land B}$}
        \RightLabel{$(\vdash\land)$}
        \BinaryInfC{$\vdash (A \lor B) \land \overline{A \land B}$}
        \UnaryInfC{$\vdash A \oplus B$}
      \end{prooftree}
      Антецедент:
      \begin{prooftree}
        \AxiomC{$A \vdash A$}
        \AxiomC{$A \vdash B$}
        \RightLabel{$(\vdash\land)$}
        \BinaryInfC{$A \vdash A \land B$}
        \AxiomC{$B \vdash A$}
        \AxiomC{$B \vdash B$}
        \RightLabel{$(\vdash\land)$}
        \BinaryInfC{$B \vdash A \land B$}
        \RightLabel{$(\lor\vdash)$}
        \BinaryInfC{$A \lor B \vdash A \land B$}
        \RightLabel{$(\neg\vdash)$}
        \UnaryInfC{$A \lor B, \overline{A \land B} \vdash$}
        \RightLabel{$(\land\vdash)$}
        \UnaryInfC{$(A \lor B) \land \overline{A \land B} \vdash$}
        \UnaryInfC{$A \oplus B \vdash$}
      \end{prooftree}
      Итоговые правила:
      \begin{prooftree}
        \AxiomC{$A, B \vdash B$}
        \AxiomC{$\vdash A, B$}
        \BinaryInfC{$\vdash A \oplus B$}
      \end{prooftree}
      \begin{prooftree}
        \AxiomC{$A \vdash B$}
        \AxiomC{$B \vdash A$}
        \BinaryInfC{$A \oplus B \vdash$}
      \end{prooftree}
    \end{solution}
  \end{itemize}
\end{enumerate}
\clearpage
