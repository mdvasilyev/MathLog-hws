\section{Многозначные логики}
\begin{enumerate}
  \item (3 балла) В какой из многозначных логик, рассмотренных выше, истинны все пропозициональные тавтологии? Обоснуйте ваш ответ.
  \begin{solution}
    все изи
    \begin{displaymath}
      \begin{array}{c c|c}
      p & q & p \land q\\
      \hline
      T & T & T\\
      T & F & F\\
      F & T & F\\
      F & F & F\\
      \end{array}
      \end{displaymath}
  \end{solution}
  \item Определите связку $\leftrightarrow$ для следующих логик:
  \begin{enumerate}
    \item (1 балл) сильная логика Клини, таблица истинности
    \begin{solution}
      По определению $A \leftrightarrow B = (A \rightarrow B) \land (B \rightarrow A)$.
      \begin{displaymath}
      \begin{array}{c c|c|c|c}
        A & B & A \rightarrow B & B \rightarrow A & A \leftrightarrow B\\
        \hline
        T & T & T & T & T\\
        T & I & I & T & I\\
        T & F & F & T & F\\
        I & T & T & I & I\\
        I & I & I & I & I\\
        I & F & I & T & I\\
        F & T & T & F & F\\
        F & I & T & I & I\\
        F & F & T & T & T
      \end{array}
      \end{displaymath}
    \end{solution}
    \item (1 балл) слабая логика Клини, таблица истинности
    \begin{solution}
      \begin{displaymath}
      \begin{array}{c c|c|c|c}
        A & B & A \rightarrow B & B \rightarrow A & A \leftrightarrow B\\
        \hline
        T & T & T & T & T\\
        T & I & I & I & I\\
        T & F & F & T & F\\
        I & T & I & I & I\\
        I & I & I & I & I\\
        I & F & I & I & I\\
        F & T & T & F & F\\
        F & I & I & I & I\\
        F & F & T & T & T
      \end{array}
      \end{displaymath}
    \end{solution}
    \item (1 балл) логика Приста, таблица истинности
    \begin{solution}
      \begin{displaymath}
      \begin{array}{c c|c|c|c}
        A & B & A \rightarrow B & B \rightarrow A & A \leftrightarrow B\\
        \hline
        T & T & T & T & T\\
        T & I & I & T & I\\
        T & F & F & T & F\\
        I & T & T & I & I\\
        I & I & I & I & I\\
        I & F & I & T & I\\
        F & T & T & F & F\\
        F & I & T & I & I\\
        F & F & T & T & T
      \end{array}
      \end{displaymath}
    \end{solution}
    \item (1 балл) логика Гёделя, кусочно-заданная функция
    \begin{solution}
      Возможные значения:
      \begin{equation}
        A_i = 0, \frac{1}{k-1}, \frac{2}{k-1}, \dots, \frac{k-2}{k-1}, 1.
      \end{equation}
      При проходе по множеству в сторону увеличения $A_i$ будем получать случай $A_i\leq A_{i+1}$, поэтому функция $A\rightarrow B$ будет во всех точках равна 1.
      \begin{equation}
        A\rightarrow B = 1.
      \end{equation}
      В случае $B\rightarrow A$ случаи в импликации будут другими, будет выбираться значения $B$, поэтому функция будет следующей:
      \begin{equation}
        B\rightarrow A =
        \begin{cases}
          0, & [0, \frac{1}{k-1}] \\
          \frac{1}{k-1}, & (\frac{1}{k-1}, \frac{2}{k-1}] \\
          \dots \\
          \frac{k-2}{k-1}, & (\frac{k-3}{k-1}, \frac{k-2}{k-1}] \\
          1, & (\frac{k-2}{k-1}, 1]
        \end{cases}
      \end{equation}
      Так как конъюнкция определяется как минимум, то итоговая функция будет аналогичной:
      \begin{equation}
        B\leftrightarrow A =
        \begin{cases}
          0, & [0, \frac{1}{k-1}] \\
          \frac{1}{k-1}, & (\frac{1}{k-1}, \frac{2}{k-1}] \\
          \dots \\
          \frac{k-2}{k-1}, & (\frac{k-3}{k-1}, \frac{k-2}{k-1}] \\
          1, & (\frac{k-2}{k-1}, 1]
        \end{cases}
      \end{equation}
    \end{solution}
  \end{enumerate}
  \item (3 балла) Рассмотрим функцию голосования $maj(x1, ..., xn)$. Она обращается в единицу, когда среди поданных на вход значений, единиц не меньше, чем нулей. В остальных случаях она обращается в ноль. Выпишите полином Жегалкина методом неопределённых коэффициентов для функции $maj(x1, x2, x3)$.
\end{enumerate}

\clearpage
